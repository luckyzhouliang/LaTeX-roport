\chapter{基于hp自适应伪谱法的再入轨迹设计与分析}

飞船再入环境严苛,整个再入段受热流、过载、动压等约束的影响。应急条件下,飞船为到达目标着陆场,再入轨迹存在
不同的轨迹形式。本章研究飞船再入机动可达域的快速确定问题,利用伪谱方法完成不同再入轨迹形式下的可达域确定。
通过容许不同的轨迹形式,扩展再入点的范围,为应急返回提供更大的窗口。

\section{再入返回轨迹设计问题建模}
\subsection{再入动力学方程}
在探月飞船再入返回轨迹设计问题中,采用三自由度运动模型,采用“瞬时平衡假设”,认为返回器始终以配平攻角飞行,且侧滑角为零。
考虑地球自转,圆球假设下,忽略地球扁率和侧滑角等的影响,在半速度系下建立运动方程,以时间为自变量的飞船无动力再入动力学方程满足
\begin{equation}
	\left\{
	\begin{aligned}
		\frac{\dif r}{\dif t}=       & v \sin \theta                                                           \\
		\frac{\dif \lambda}{\dif t}= & \frac{v \cos \theta \sin \psi}
		{r \cos \phi}                                                                                          \\
		\frac{\dif \phi}{\dif t}=    & \frac{v \cos \theta \cos \psi}{r}                                       \\
		\frac{\dif v}{\dif t}=       & -D-g \sin \theta +\omega_e^2r\cos\phi \sin\theta \cos\phi-
		\omega_e^2r\cos\phi\cos\theta\sin\phi\cos\sigma                                                        \\
		\frac{\dif\theta}{\dif t} =  & \frac{L\cos\sigma}{v}+(\frac{v}{r}-
		\frac{g}{v})\cos\theta+2\omega_e\cos\phi\sin\sigma                                                     \\
		                             & {}+\frac{\omega_e^2r\cos\phi\cos\theta\cos\phi}{v}                      \\
		\frac{\dif\psi}{\dif t} =    & \frac{L\sin\sigma}{v\cos\theta}+\frac{v\sin \psi\cos \theta\tan\phi}{r}
		-2\omega_e(\cos\phi\tan\theta\cos\sigma-\sin\phi)                                                      \\
		                             & {}+\frac{\omega_e^2r}{\cos\theta}sin\phi\cos\phi\sin\sigma
	\end{aligned}
	\right.
\end{equation}
其中,$r$为地心距,$\lambda$和$\phi$为经纬度,$v$为相对地球速度,$\theta$为当地速度倾角,表征速度与当地水平的夹角,
以及$\psi$为速度方位角,表征速度在当地水平面内的投影与当地正北方向的夹角,顺时针为正;% 表述可能错误
$\sigma$为控制量倾侧角,反映了升力对铅锤面的倾斜;$ \omega_e $为地球自转角速度,$L$,$D$为再入过程中的升力阻力加速度,满足
\begin{align}
	L & =\frac{C_L\rho v^2S}{2m} \\
	D & =\frac{C_D\rho v^2S}{2m}
\end{align}
其中:$ C_L $和$ C_D $为气动升力系数和气动阻力系数,$ \rho $为大气密度,$ m $为飞行器质量,$ S $为飞船的参考面积。

大气密度$ \rho $可采用指数模型、多阶段拟合公式和美国US1976标准大气模型等,考虑到精度和计算速度的要求,最终
采用多阶段的拟合公式,更多可参考\upcite{贾沛然-1993-远程火箭弹道学}。
\begin{equation}
	\rho=\rho_0 e^{-\beta h}
\end{equation}

\subsection{约束分析}
因探月返回器高速再入的特点,过载和热流等约束十分苛刻。为了保证再入过程的安全,根据约束的性质划分为两类:过程约束和终端约束。
\subsubsection{过程约束}
\begin{enumerate}
	\item 热流约束\par
	      高超声速气动加热对热防护系统(TPS)的影响,为避免飞行器被烧毁,因此需对飞行器再入过程中的热流密度进行限制。
	      通常采用驻点热流密度作为约束指标,计算公式如下:
	      \begin{equation}
		      \dot{Q} = k_s\left(\frac{\rho}{\rho_{0}}\right)^{0.5}\left(\frac{v}{V_{c}}\right)^{m}
	      \end{equation}
	      其中,$ k_s $为热流相关常数,$ V_c=7.9\mathrm{km/s} $为参考的第一宇宙速度,$ m $可取3或3.15,本文中取3.15
	\item 过载约束\par
	      考虑到内部结构、材料的限制,需要对过载进行限制,要求瞬时过载小于最大允许过载,即
	      \begin{equation}
		      n=\sqrt{{{L}^{2}}+{{D}^{2}}}\le {{n}_{\max }}
	      \end{equation}
	      其中,$ n_{\max} $为最大允许过载值。
	\item 动压约束\par
	      动压对控制系统的影响和侧向稳定性的要求,
	      \begin{equation}
		      q=\frac{1}{2} \rho V^{2} \leq q_{\max }
	      \end{equation}
	\item 跃起高度约束\par
	      此外,为防止跳出高度过高导致任务失败,路径约束还应包括一个高度约束。这里设定最大高度不超过300km,
	      \begin{equation}
		      h<h_{\max}
	      \end{equation}
	\item 控制量约束\par
	      控制量倾侧角受限于执行机构,飞船的控制能力有限,实际倾侧角机动不可能瞬时完成,需要对倾侧角和倾侧角变化
	      速率进行限幅,即
	      \begin{equation}
		      \left\{
		      \begin{aligned}
			      \abs{\sigma}       & \leq \sigma_{\max}       \\
			      \abs{\dot{\sigma}} & \leq \dot{\sigma}_{\max} \\
		      \end{aligned} \right.
	      \end{equation}
\end{enumerate}

\subsubsection{终端约束}
对于再入问题中的终端约束,包含起始点和终点的状态约束。起始点处的状态在分析中通常为给定值,在初始时刻$ t_0 $状态满足
\begin{equation}
	\left\{ \begin{aligned}
		h(t_0)=h_0,\lambda(t_0)=\lambda_0,\phi(t_0)=\phi_0 \\
		v(t_0)=v_0,\theta(t_0)=\theta_0,\psi(t_0)=\psi_0
	\end{aligned} \right.
\end{equation}

终点处根据研究问题可分为固定终点和自由终点两种情况,数学上表达为:
\begin{enumerate}
	\item 固定终点的约束条件\par
	      终点处位置固定,但速度和速度倾角存在范围约束,即:
	      \begin{equation}
		      \left\{ \begin{aligned}
				  h(t_f)&=h_{f}, \lambda(t_f)=\lambda_{f}, \phi(t_f)=\phi_{f} \\
				  v &\geq v_{f}, \theta_{\min } \leq \theta \leq \theta_{\max }
		      \end{aligned}\right.
	      \end{equation}
	\item 自由终点的约束条件\par
	对于求解可达域等问题时,终点不固定,通常约束设置为:
	\begin{equation}
		h(t_f)=h_{f}, v \geq v_{f}, \theta_{\min } \leq \theta \leq \theta_{\max }
	\end{equation}
\end{enumerate}

\section{hp自适应伪谱法的求解策略}
hp自适应伪谱法是伪谱方法的一种,求解再入问题在计算速度有较大优势。hp自适应伪谱法主要包含两部分内容:伪谱化离散和hp自适应网格细化


\section{伪谱法快速可达域计算与分析}
\section{短航程再入轨迹设计结果对比}

\subsection{}